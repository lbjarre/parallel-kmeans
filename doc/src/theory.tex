%% Theory section

\newcommand{\dset}{\mathcal{X}}
\newcommand{\dpoint}{\bm{x}}
\newcommand{\mean}{\bm{\mu}}
\newcommand{\meann}{\bar{\bm{\mu}}}

\newcommand{\cls}{\text{class}}
\newcommand{\cnt}{\text{count}}

\subsection{$k$-means clustering}

$k$-means clustering is a data clustering method
which clusters input data from the data set $\dset$ into $k$ different classes.
The classes are represented by the class means $\mean_i$ and points are considered to be in a class $S_i$
if the squared distance to the class mean is the minimum
compared to the squared distance to the other class means.
Formally:
\[
      S_i = \{ \dpoint \in \dset
   :  \| \dpoint - \mean_i \|^2 \leq \| \dpoint - \mean_j \|^2
  ,\, \forall 1 \leq j \leq k \}
\]
A clustering method aims to find a selection of these classes
$\mathcal{S} = \{S_1, S_2, \ldots, S_k\}$
which divides the data points in some favorable way.
$k$-means finds the placement of the class means
by minimization of the summed squared distance of all class points to the class mean for all $k$ classes:
\[
  \mathcal{S}_{k\text{-means}}
  = \argmin_\mathcal{S} \sum_{i = 1}^k \sum_{\dpoint \in S_i} \| \dpoint - \mean_i \|^2
\]
A common algorithm to find this is Lloyd's algorithm \cite{lloyd},
which iteratively classifies points according to current class means
and updates them with the average of all classified points until convergence.
\Cref{alg:lloyds} describes this procedure in pseudocode.
This algorithm requires multiple loops for each iteration.
The most costly part is the classification of all data points,
which requires computation on all $d$ dimensions of the data set
for all $k$ classes over all $n$ points,
giving each iteration a complexity of $O(nkd)$.

\begin{algorithm}[!ht]

  \SetKw{Init}{initialize}

  \BlankLine
  \KwIn{Data points $\dset = \{ \dpoint_1, \dpoint_2, \ldots, \dpoint_n\}$
        with $\dpoint_i \in \mathbb{R}^d \; \forall \dpoint_i \in \dset$,
        number of clusters $k$}
  \KwOut{Class means $\mean_1, \mean_2, \ldots, \mean_k$}
  \BlankLine

  \Init $\meann_i$ on random points in $\dset \; \forall i = 1, \ldots, k$ \;
  \While{$\mean_i \neq \meann_i \; \forall i = 1, \ldots, k$}{

    \For{$i = 1, \ldots, k$}{

      $\mean_i \gets \meann_i$ \;
      $\meann_i \gets 0$ \;

    }

    \ForAll{$\dpoint \in \dset$}{

      $\cls \gets \argmin_k \| \dpoint - \mean_k \|^2$ \;
      $\cnt_\cls \gets \cnt_\cls + 1$ \;
      $\meann_\cls \gets \meann_\cls + \dpoint$ \;
    
    }
    
    \For{$i = 1, \ldots, k$}{
      $\meann_i \gets \frac{\meann_i}{\cnt_i}$ \;
      $\cnt_i \gets 0$ \;
    }
  
  }
  
  \Return{$\{ \mean_1, \mean_2, \ldots, \mean_k \}$}

  \caption{Lloyd's algorithm for finding the $k$-means clustering class means.}
  \label{alg:lloyds}

\end{algorithm}

\subsection{Parallelisation strategy}

\newcommand{\Titer}{T_{\text{iter}}}
\newcommand{\tcalc}{t_{\text{calc}}}
\newcommand{\tcomm}{t_{\text{comm}}}

\Cref{alg:lloyds} shows a lot of possibility of parallelisation
due to the large number of independent calculations.
For every iteration all the points needs to be reclassified to the new means,
for which every point is independent of the other points.
Computing the new class means consist of adding up the classified points over all $d$ dimensions
as well as the total count,
which are additions that can be done firstly on separate processors and finally reduced over all processors.

Overall we get a execution time for each iteration $\Titer$ that with the number of processes $P$
that roughly scales as
\[
  \Titer = \tcalc\frac{nkd}{P} + \tcomm k(d + 1)\log P
\]
Noteworthy for the analysis of this performance is that $n$ will be much larger than $k$ and $d$
in most applications.
$k$ and $d$ normally lie in the $10^1$-$10^2$ range,
while $n$ can go up to $10^6$-$10^9$ easily.
The reduction of the calculation time will therefore affect the overall computation time
more than the increasing communication time.

\Cref{alg:parlloyds} shows the minor change to make \cref{alg:lloyds} into a parallel algorithm.


\begin{algorithm}[!ht]

  \newcommand{\reduce}{\text{reduce\_sum}}
  
  \SetKw{Init}{initialize}

  \BlankLine
  \KwIn{Data points $\dset = \{ \dpoint_1, \dpoint_2, \ldots, \dpoint_n\}$
        with $\dpoint_i \in \mathbb{R}^d \; \forall \dpoint_i \in \dset$,
        number of clusters $k$,
        number of processes $P$}
  \KwOut{Class means $\mean_1, \mean_2, \ldots, \mean_k$}
  \BlankLine

  Distribute dataset into local datasets $\dset_i \; \forall i = 1, \ldots, P$

  \Init $\mean_i$ on random points in $\dset \; \forall i = 1, \ldots, k$ \;

  \While{$\mean_i \neq \meann_i \; \forall i = 1, \ldots, k$}{

    \For{$i = 1, \ldots, k$}{
      
      $\mean_i \gets \meann_i$ \;
      $\meann_i \gets 0$ \;

    }

    \ForAll{$\dpoint \in \dset_i$}{

      $\cls \gets \argmin_k \| \dpoint - \mean_k \|^2$ \;
      $\cnt_\cls \gets \cnt_\cls + 1$ \;
      $\meann_\cls \gets \meann_\cls + \dpoint$ \;

    }

    \For{$i = 1, \ldots, k$}{

      $\reduce(\meann_i)$ \;
      $\reduce(\cnt_i)$ \;

      $\meann_i \gets \frac{\meann_i}{\cnt_i}$ \;
      $\cnt_i \gets 0$ \;

    }

  }

  \Return{$\{ \mean_1, \mean_2, \ldots, \mean_k \}$}

  \caption{Parallel version of Lloyd's algorihtm}
  \label{alg:parlloyds}

\end{algorithm}

