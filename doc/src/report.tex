\documentclass{article}

% Preamble file

\usepackage[utf8]{inputenc}
\usepackage[T1]{fontenc}

\usepackage{geometry}

\usepackage{mathtools}
\usepackage{amsfonts}
\usepackage{bm}
\usepackage[linesnumbered, ruled, noend]{algorithm2e}

\usepackage[hidelinks]{hyperref}
\usepackage{cleveref}

\usepackage{graphicx}
\usepackage{subcaption}
\usepackage{booktabs}

\usepackage{tikz}
\usepackage{pgfplots}
\usepackage{pgfplotstable}
\usepgfplotslibrary{fillbetween}

\geometry{
  top=2cm,
  bottom=2cm,
  left=3.5cm,
  right=3.5cm
}

\bibliographystyle{plain}

\newcommand{\mail}[1]{\href{mailto:#1}{#1}}

\DeclareMathOperator*{\argmin}{arg\,min}

\DontPrintSemicolon

\pgfplotsset{
  compat=1.13,
  every axis/.prefix style={
    width=\textwidth,
    height=.8\textwidth,
    legend cell align=left
  },
  every axis plot/.style={
    no marks,
    line width=.8pt
  }
}


\begin{document}

\section{$k$-means clustering}
$k$-means clustering is a data clustering method which clusters input data into $k$ different classes.
The classes are represented by the class means $\mu_i$ and points are considered to be in a class $S_i$
if the squared distance to the class mean is the minimum
compared to the squared distance to the other class means.
Formally:
\[
  S_i = \{ x \in X : ||x - \mu_i||^2 \leq ||x - \mu_j||^2 \forall 1 \leq j \leq k \}
\]
$k$-means finds the placement of the class means
by minimization of the summed squared distance of all class points to the class mean for all $k$ classes:
\[
  \argmin_S \sum_{i=1}^k \sum_{x \in S_i} ||x - \mu_i||^2
\]
A common algorithm to find this is Lloyd's algorithm,
which iteratively classifies points according to current class means
and updates them with the average of all classified points until convergence.

\end{document}
